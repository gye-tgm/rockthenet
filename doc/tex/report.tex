\documentclass[11pt, a4paper]{article}
% \usepackage[T1]{fontenc}
\usepackage[utf8]{inputenc}
\usepackage{listings}
\usepackage[margin=1.0in]{geometry}
\usepackage{color}
\usepackage{graphicx}
\usepackage{tabularx}
\usepackage{url} 

\title{Rock the net}
\author{Elias Frantar, Samuel Schmidt, Nikolaus Schrack, Gary Ye}
\date{\today{}, Wien}
\begin{document}

\lstset{ %
  backgroundcolor=\color{white},   % choose the background color; you must add \usepackage{color} or \usepackage{xcolor}
  basicstyle=\footnotesize,        % the size of the fonts that are used for the code
  breakatwhitespace=false,         % sets if automatic breaks should only happen at whitespace
  breaklines=true,                 % sets automatic line breaking
  captionpos=b,                    % sets the caption-position to bottom
% commentstyle=\color{mygreen},    % comment style
  deletekeywords={...},            % if you want to delete keywords from the given language
  escapeinside={\%*}{*)},          % if you want to add LaTeX within your code
  extendedchars=true,              % lets you use non-ASCII characters; for 8-bits encodings only, does not work with UTF-8
% frame=single,                    % adds a frame around the code
  keepspaces=true,                 % keeps spaces in text, useful for keeping indentation of code (possibly needs columns=flexible)
% keywordstyle=\color{blue},       % keyword style
% language=bash,                   % the language of the code
  morekeywords={*,...},            % if you want to add more keywords to the set
  numbers=left,                    % where to put the line-numbers; possible values are (none, left, right)
  numbersep=5pt,                   % how far the line-numbers are from the code
  rulecolor=\color{black},         % if not set, the frame-color may be changed on line-breaks within not-black text (e.g. comments (green here))
  showspaces=false,                % show spaces everywhere adding particular underscores; it overrides 'showstringspaces'
  showstringspaces=false,          % underline spaces within strings only
  showtabs=false,                  % show tabs within strings adding particular underscores
  stepnumber=1,                    % the step between two line-numbers. If it's 1, each line will be numbered
  tabsize=2,                       % sets default tabsize to 2 spaces
  title=\lstname                   % show the filename of files included with \lstinputlisting; also try caption instead of title
}


\maketitle
\newpage
\tableofcontents
\newpage

\section{Task description}
% TODO
\section{Design}
% TODO
\includegraphics[width=\textwidth]{images/uml}
 
\section{Effort estimation}

\subsection{Basic Tasks}
\begin{tabular} {| l | l | l | l |} \hline
Task &	Original Estimate & Remaining Estimate & Time spent \\ \hline
Preparation for the Tasks &	20 &	0.00 & 24.50 \\ \hline
Listing firewall rules &	7 &	2 &	9.50 \\ \hline
Refreshing rules &	6 &	6 &	0.00 \\ \hline
Visualize thru put & 8 &	4 &	5.00 \\ \hline
Encapsulate the data retrieval &	4	& 0.00 &	4.00 \\ \hline
GUI	 & 10 &	7 &	8.00 \\ \hline
Final Documentation  &	8	& 2	& 5.00 \\ \hline
Basic Total	& 63 &	31.00 &	56.00 \\ \hline
\end{tabular}
\subsection{Advanced Tasks}
\begin{tabular} {| l | l | l | l |}\hline
Task &	Original Estimate & Remaining Estimate & Time spent \\ \hline
Alarm the user &	7	& 5	& 0 \\ \hline
Firewall rules CRUD &	6 &	8 & 	0 \\ \hline
Transactions by Multicast &	8 &	15 &	0 \\ \hline
Exchangeable & 5 & 0	 & 2 \\ \hline
Advanced Total & 26 &	28 &	2 \\ \hline
\end{tabular}

\subsection{Total}
\begin{description}
	\item[Original Estimate]: 89
	\item[Remaining Estimate]: 31
	\item[Time Spent]: 58
\end{description}

\section{Installation}

\section{Technologies}
\subsection{Mock-Objects}
\subsubsection{Setup}

\begin{enumerate}
	\item Download \textit{Mockito} from [1]
	\item Add \textit{junit-4.11.jar} and \textit{mockito-all-1.9.5.jar} to the project's \textit{classpath}
	\item Done
\end{enumerate}

\subsubsection{How to use Mockito}

\textit{Mockito} allows you to mock interfaces, but also concrete classes, with a single line of code:
	
\begin{lstlisting} 
List mockedList = mock(List.class); // creates a mock-Object of `List` 
\end{lstlisting}
	
A \textit{Mockito-Mock-Objects} remembers all methods, which have been called. So you can check afterwards if some method has been called with some parameters.
	
\begin{lstlisting}
	mockedList.add("one");
	    
	verify(mockedList).add("one"); // test "successful" because that exact method with that exact parameter has been called before
	verify(mockedList).add("two"); // test "failed" because `.add("two")` has not been called before
\end{lstlisting}

One of the main functionalities of a Mock-object is that it can provide kind of dummy-methods, called *stubs*, i.e. when a specific method with a specific parameter is called, a specific
value is returned. There are also some kind of wildcards for the methods parameters, if you want to return "first" on any given Integer-parameter. That can be achieved like this:

\begin{lstlisting}
    when(mockedList.get(0)).thenReturn("first"); // `mockedList` will return "first" when `.get(0)` is called
    System.out.println(mockedList.get(0)); // prints "first"
    
    when(mockedList.get(anyInt()).thenReturn("any int"); // will return "any int" if you pass for example: 1, 27, 4, ...
    System.out.println(mockedList.get(2345)); // prints "any int"
\end{lstlisting}

You can also make the mock throw Exceptions by using `.thenThrow()` or make void methods throw Exceptions by:

\begin{lstlisting}
    doThrow(new RuntimeException()).when(mockedList).clear();
    mockedList.clear(); // throws RunTime-Exception
\end{lstlisting}

You can also verify how often a specific method has been called. That works like this:

\begin{lstlisting}
    mockedList.add("once");
    
    verify(mockedList, times(1)).add("once"); // success because `.add("once")` has been called exactly once
    verify(mockedList, times(2)).add("once"); // fails because it has only been called once
\end{lstlisting}

For \textit{times(0)}, you should better use \textit{never()}.
 
Like the number of invocations, you can also verify the order of of invocations:

\begin{lstlisting}
    mockedList.add("first");
    mockedList.add("second");
    
    InOrder inOrder = inOrder(mockedList);
    
    inOrder.verify(mockedList).add("first");
    inOrder.verify(mockedList).add("second");  
\end{lstlisting} 
    
For explanation of additional functionalities and full documentation, see [2].

\subsection{Java FX}
Java 8 supports Java FX, so no further installation will be needed if Java 8 is used. 

\subsubsection{Eclipse}
In Eclipse kann das Addon zum Erstellen von JavaFx Applikationen unter dem Punkt “Help” -> “Install New Software” installiert werden. Der Name ist e(fx)clipse und der Link % http://download.eclipse.org/efxclipse/updates-released/1.0.0/site. 
Jetzt kann über “New” -> “Other” -> “JavaFX Project” ein Projekt erstellt werden. 

\subsubsection{IntelliJ}
Tutorial with Java 8 and IntelliJ IDEA 13+
% http://docs.oracle.com/javase/8/scene-builder-2/work-with-java-ides/sb-with-intellij.htm 
Getting Started with JavaFX
% http://docs.oracle.com/javase/8/javafx/get-started-tutorial/jfx-overview.htm 

\subsubsection{IntelliJ - TGM Licenses}
Run IntelliJ IDEA and follow the Installation Wizard's instructions. To register for use of the software or change your existing registration details, go to Help/Register menu of the software and enter the included below the User Name and License Key(s) into the registration dialog:

User Name: TGM - Institute of Technology 
License Key: 603253-X846L-X51EC-FMQQ7-EXY9B-DO323

\subsection{Charts}
Für unsers Applikation wird ein Line Chart gebraucht, was mit JavaFX möglich ist. 
% http://docs.oracle.com/javafx/2/charts/line-chart.htm#CIHGBCFI

\subsection{SNMP - Simple Network Management Protocol}

It is, like the name says, a simple network protocol to manage and monitor the devices in a network. Practically, the monitored devices are all connected to a central system that controls them by using the “Simple Network Manager Protocol”.

Figure 1: the Router, the Server, the Switch and the Printer are connected via the SNMP to the management system.
% (Source: http://de.wikipedia.org/wiki/Simple_Network_Management_Protocol)

\subsubsection{SNMPv2c and SNMPv3 differences}

With v3 no changes to the protocol aside from added security and remote configuration enhancements (we use telnet anyway) to SNMP were added. However new textual conventions, concepts and terminology were introduced.

The core Snmp class documentation[3] offers only insight on big differences in SNMPv1(asynchronous) and v3(synchronous) implementations. TransportMapping only affects UDP/TCP/… configurations.

\subsection{MIB - Management Information Base}
As the management system wanting to read data from a device, it has to know the data of the device. The specified informations can be retrieved from the corresponding MIB. 

The MIB of the device, that is being used for this exercise, can be found here [1].

The MIB is structured like a tree (examples can be found in the Wikipedia article). Every object of the device can be identified by the OID (Object Identifier), which is just a list of numbers separated by a dot.

One can for example GET/SET the variables by using the OI of the specified object.

Also one can use a MIB browser to explore the structure [2].

[1] http://www.oidview.com/mibs/2636/JUNIPER-NSM-TRAPS.html
[2] http://ireasoning.com/mibbrowser.shtml 
[3] http://www.snmp4j.org/doc/org/snmp4j/Snmp.html 
\subsubsection{Connection}
\subsubsection{MIBs}
\subsection{Multicast-Groups}

Multicast is basically a group communication where the packages are sent only once and arrive at their destination simultaneously. The nodes of the network (router, switches) will take care of how the messages will be distributed. So the sender does not have to worry about the number of receivers.

The most common protocol for this is UDP, even though it’s unreliable. There exists also reliable protocols where loss detection and retransmission are provided.


In the Java API there exists a class to provide multicasting [1]. The documentation itself says:
“A multicast group is specified by a class D IP address and by a standard UDP port number. Class D IP addresses are in the range 224.0.0.0 to 239.255.255.255, inclusive. The address 224.0.0.0 is reserved and should not be used.”

A code snippet has also been provided:

\begin{lstlisting}
// join a Multicast group and send the group salutations
...
String msg = "Hello";
InetAddress group = InetAddress.getByName("228.5.6.7");
MulticastSocket s = new MulticastSocket(6789);
s.joinGroup(group);
DatagramPacket hi = new DatagramPacket(msg.getBytes(), msg.length(),
group, 6789);
s.send(hi);
// get their responses!
byte[] buf = new byte[1000];
DatagramPacket recv = new DatagramPacket(buf, buf.length);
s.receive(recv);
...
// OK, I'm done talking - leave the group...
s.leaveGroup(group);
\end{lstlisting}



The important methods of the MulticastSocket class are:
joinGroup: to join the given group
receive: receive the message of the subscribed group
leaveGroup: leave the specified group



[1] http://docs.oracle.com/javase/7/docs/api/java/net/MulticastSocket.html

\section{User Stories}
 \textbf{Basic tasks:}
\begin{itemize}
\item As a user, I want to connect to my firewall device by entering its IP-address.
\begin{itemize}
\item As a user, I want to connect with SNMPv3.
\item As a user, I want to fallback to SNMPv2c when SNMPv3 is not available.
\end{itemize}
\item As a user, I want a visually appealing graphical user interface.
\item As a user, I want to list all firewall-rules configured on my device.
\begin{itemize}
\item As a user, I want to see the port of a firewall rule.
\item As a user, I want to see the zone of a firewall rule.
\item As a user, I want to see the service of a firewall rule.
\item As a user, I only want to see the TCP and UDP firewall rules.
\end{itemize}
\item As a user, I want the rules to be refreshed.
\begin{itemize}
\item As a user, I want the rules to be refreshed automatically.
\begin{itemize}
\item As a user, I want to configure the refresh intervals.
\end{itemize}
\item As a user, I want to refresh the rules manually by pressing on a button.
\end{itemize}
\item As a user, I want to visually monitor the thru-put of a firewall-rules.
\begin{itemize}
\item As a user, I want to visually see the thru-put in line-chart.
\item As a user, I want to monitor multiple rules at the same time.
\end{itemize}
\item As a developer, I want the data retrieval to be encapsulated for further reuse and expansion.
\item As a developer, I want all network packages to be logged in a log file for debugging purposes.
\end{itemize}
 \textbf{Advanced tasks:}
\begin{itemize}
\item As a user, I want to be notified if a rule has been changed without my assistance.
\begin{itemize}
\item As a user, I want to be notified by a pop-up window.
\item As a user, I want to be optionally notified by email.
\item As a user, I want to configure my email-address.
\begin{itemize}
\item As a user, I want to configure the email-address in a config-file.
\item As a user, I want to configure the email-address in the GUI.
\end{itemize}
\end{itemize}
\item As a user, I want to configure all firewall-rules on the device.
\begin{itemize}
\item As a user, I want to create a fully new rule.
\item As a user, I want to fully edit an existing rule.
\item As a user, I want to delete an existing rule.
\item As a user, I want to login before editing a rule.
\begin{itemize}
\item As a user, I want to stay logged in for the whole session after I have logged in for the first time.
\end{itemize}
\end{itemize}
\item As a user, I want to have a transaction system for write access on the firewall, so that no editing conflicts can occur when multiple applications are trying to modify at the same time.
\begin{itemize}
\item ???
\item ???
\item ...
\end{itemize}
\item As a developer, I want to be able to easily modify the software so that it works with other firewall hardwares.
\begin{itemize}
\item As a developer, I only want to exchange the SNMP-commands to make the application work with other firewall hardwares.
\end{itemize}
\end{itemize}
\section{Test report}
\section{Occurred problems}
\bibliography{protokoll}{}
\bibliographystyle{plain}
\end{document}
